\documentclass[10pt,a4paper]{article}
\usepackage[utf8]{inputenc}
\usepackage{amsmath}
\usepackage{amsfonts}
\usepackage{amssymb}
\usepackage{hyperref}

\author{David Rozas, Nigel Gilbert, Paul Hodkinson}
\title{Contribution beyond source code in Free/Libre Open Source Software: the role of affective labour in the Drupal community}
\begin{document}

\maketitle

\texttt{}
\section{Abstract (250w max)}
Contribution is a key element of Commons-Based Peer Production (CBPP) communities. This element becomes of even more relevance for those communities focused on the production of digital commons, which typically possess the characteristics of an economy of contribution, rather than an economy of gift (Wittel, 2013), as in the case of the Free/Libre Open Source (FLOSS) communities. Nevertheless, most of the literature on FLOSS has focused its attention on the most visible outcome of the contribution: the collaboratively built shared objects: source code, documentation, translations, user support, etc. However, less attention has been paid to those collaborative activities which Hardt (1999) defines as affective labour, referring to the  immaterial labour present in human interaction which produces or modifies emotional experiences, including intangible assets, such as excitement, passion or the sense of community which have been identified as contribution motivators in FLOSS communities.

The goal of this study is to understand what kind of activities are perceived as contributions in the Drupal community, by carrying out qualitative research that could help to shed light on those other activities that have not been widely studied due to their lack of visibility. This aspect is specially critical in a community that has been characterised as "code-centric" (Zilouchian, 2011; Sims, 2013). We aim to analyse how the whole set of identified contribution activities are perceived and evaluated by the members of the community, as well as their representation or lack of it in the community's digital collaboration platform.

\section{If your paper is based on current research, please indicate the stage the research has reached and the methodology used:}

This study is part of the initial stage of my PhD research. In order to understand how  the Drupal community works, it is necessary to understand first what kind of activities are perceived as contributions. I am following a virtual ethnographic approach (Hine, 2000), and my sources of data are: field notes from participant-observation, transcriptions from semi-structured qualitative interviews and content analysis from online sources (e.g.: Drupal Planet)

\section{If you have published related work from this research, please provide references:}

David Rozas. 2014. Drupal as a Commons-Based Peer Production community: a sociological perspective. In Proceedings of The International Symposium on Open Collaboration (OpenSym '14). ACM, New York, NY, USA, , Pages 36 , 2 pages. DOI=10.1145/2641580.2641624 http://doi.acm.org/10.1145/2641580.2641624


\section{If this research has particular policy/practice implications, please indicate these:}

N/A

\end{document}